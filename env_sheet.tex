%&context
%%
%% Kanji and Hanzi writing practice sheets
%% Environment definition


\startenvironment env_sheet

\setuplayout[top=1cm,header=0pt,bottom=1cm,footer=20pt]

\usemodule[simplefonts]

%% enter font to be used here
%\setmainfont[stkaiti]
%\setmainfont[ukaicn]

\setupbodyfont[base, 12pt]
\setuppagenumber[state=start]
%\setupindenting[none]
\setuppagenumbering[location=footer]


%% Change for different size:
\def\symbolsize{15mm}
\def\readingsize{4mm}

\definefont[bigfont][name:stkaiti at \symbolsize]
\definefont[descfont][name:stkaiti at \readingsize]
\definefont[readfont][RegularBold at \readingsize]
\definefont[translfont][Regular at \readingsize]


\startuseMPgraphic{line}
  pickup pencircle scaled .7pt;
  picture c, box;
  c=\sometxt{\MPvar{c}};
  color grey;
  grey=0.8*white;
  draw unitsquare scaled \symbolsize\space withcolor 0.25white;
  pickup pencircle scaled 0.5pt;
  draw .5[llcorner currentpicture,lrcorner currentpicture] --
    .5[ulcorner currentpicture,urcorner currentpicture]
    withcolor grey;
  draw .5[llcorner currentpicture,ulcorner currentpicture] --
    .5[lrcorner currentpicture,urcorner currentpicture]
    withcolor grey;
  draw llcorner currentpicture --
    urcorner currentpicture
    withcolor grey;
  draw ulcorner currentpicture --
    lrcorner currentpicture
    withcolor grey;
  box := currentpicture;
  setbounds box to boundingbox box enlarged -1pt;
  currentpicture := nullpicture;

  numeric imax;
  imax = 180mm/bbwidth(box);
  draw box shifted (1*bbwidth(box), 0pt);
  draw c shifted ((center box)-(center c) + (1*bbwidth(box), 0pt))
    withcolor 0.1*white;
  for i = 2 step 1 until imax:
    %if i = 1:
    %  draw c shifted ((center box)-(center c) + (i*bbwidth(box), 0pt))
    %    withcolor (0.5+i/(imax+15))*white;
    %fi;
    draw box shifted (i*bbwidth(box), 0pt);
  endfor;
\stopuseMPgraphic

\def\line#1{{%
  \par
  \leavevmode
  \bigfont
  \useMPgraphic{line}{c=#1}%
  \vskip 5mm
  \par
}}

\setuplines
  [before={\startframedtext[frame=off]},
   after=\stopframedtext]

\def\dolinedesc[#1][#2][#3][#4]{{%
  \iffourthargument
    \setvariables[line][description=#1]
    \setvariables[line][symbol=#2]
    \setvariables[line][reading=#3]
    \setvariables[line][translation={#4}]
  \else
    \setvariables[line][description=#1]
    \setvariables[line][symbol=#1]
    \setvariables[line][reading=#2]
    \setvariables[line][translation={#3}]
  \fi
  \startlines
  \par
  {\descfont   \getvariable{line}{description}} --
  {\readfont   \getvariable{line}{reading}} --
  {\translfont \getvariable{line}{translation}}
  \par
  \leavevmode
  \bigfont
  \useMPgraphic{line}{c=\getvariable{line}{symbol}}%
  \stoplines
  \vskip -15mm
}}

\def\linedesc{\doquadrupleargument\dolinedesc}


%\startluacode
%  userdata = userdata or {}

%  --reads a file with one hanzi per line
%  --generates TeX code for including each hanzi
%  function userdata.readhanzi(file)
%    for l in io.lines(file) do
%      tex.sprint("\\line{" .. l .. "}\n")
%    end
%  end
%\stopluacode


\usemodule[filter]

\defineexternalfilter
  [anki]
  [filtercommand={python filter.py --tex --reading \externalfilterinputfile \space
                                                   \externalfilteroutputfile}]


\stopenvironment
