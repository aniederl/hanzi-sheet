%% Kanji writing practice sheets

%% Note: This is a ConTeXt file, to be compiled using
%% texexec --xtx sheet.tex

%\usemodule[simplefonts]
%\setmainfont[hiraginokakugothicpro]
%\mainlanguage[ja]
%\language[ja]


\setuplayout[top=1cm,header=0pt,bottom=1cm,footer=0pt]

%\setupunicodefont[japanese][scale=1.0]
%\definefontsynonym [JapaneseMinchoRegular][cyberp]
%\defineunicodefont [Mincho][JapaneseMincho][japanese]

%\enableregime[utf]

\usemodule[simplefonts]
\setmainfont[stkaiti]

%% Select the font to use here (end of line):
%\definetypeface[base][rm][Xserif][STKaiti]
%\definetypeface[base][rm][Xserif][stkaiti]
\setupbodyfont[base, 12pt]
\setuppagenumber[state=stop]
\setupindenting[none]

%% Change for different size:
\def\kanjisize{15mm}
\definefont[bigfont][Regular at \kanjisize]

\startuseMPgraphic{line}
  pickup pencircle scaled .7pt;
  picture c, box;
  c=\sometxt{\MPvar{c}};
  color grey;
  grey=0.8*white;
  draw unitsquare scaled \kanjisize\space withcolor 0.25white;
  pickup pencircle scaled 0.5pt;
  draw .5[llcorner currentpicture,lrcorner currentpicture] --
    .5[ulcorner currentpicture,urcorner currentpicture]
    withcolor grey;
  draw .5[llcorner currentpicture,ulcorner currentpicture] --
    .5[lrcorner currentpicture,urcorner currentpicture]
    withcolor grey;
  draw llcorner currentpicture --
    urcorner currentpicture
    withcolor grey;
  draw ulcorner currentpicture --
    lrcorner currentpicture
    withcolor grey;
  box := currentpicture;
  setbounds box to boundingbox box enlarged -1pt;
  currentpicture := nullpicture;

  numeric imax;
  imax = 180mm/bbwidth(box);
  draw box shifted (1*bbwidth(box), 0pt);
  draw c shifted ((center box)-(center c) + (1*bbwidth(box), 0pt))
    withcolor 0.1*white;
  for i = 2 step 1 until imax:
    %if i = 1:
    %  draw c shifted ((center box)-(center c) + (i*bbwidth(box), 0pt))
    %    withcolor (0.5+i/(imax+15))*white;
    %fi;
    draw box shifted (i*bbwidth(box), 0pt);
  endfor;
\stopuseMPgraphic

\startuseMPgraphic{line2}
  pickup pencircle scaled .7pt;
  picture box;
  color grey;
  grey=0.8*white;
  draw unitsquare shifted (0,240) scaled \kanjisize\space withcolor 0.25white;
  pickup pencircle scaled 0.5pt;
  draw .5[llcorner currentpicture,lrcorner currentpicture] --
    .5[ulcorner currentpicture,urcorner currentpicture]
    withcolor grey;
  draw .5[llcorner currentpicture,ulcorner currentpicture] --
    .5[lrcorner currentpicture,urcorner currentpicture]
    withcolor grey;
  draw llcorner currentpicture --
    urcorner currentpicture
    withcolor grey;
  draw ulcorner currentpicture --
    lrcorner currentpicture
    withcolor grey;
  box := currentpicture;
  setbounds box to boundingbox box enlarged -1pt;

  numeric imax;
  imax = 180mm/bbwidth(box);
  for i = 2 step 1 until imax-1:
    draw box shifted (i*bbwidth(box), 0pt);
  endfor;
\stopuseMPgraphic



\def\line#1{{%
  \par
  \leavevmode
  %\hskip-1.75cm
  \bigfont
  \useMPgraphic{line}{c=#1}%
  %\crlf
  %\useMPgraphic{line2}%
  \crlf
  \par
}}



\starttext
\line{你}
\line{好}
\line{大}
\line{小}
\line{高}
\line{喜}
\line{欢}
\line{谁}
\line{最}
\line{近}
\line{现}
\line{在}
\line{你}

\stoptext

